\section{Introduction}
In recent years, deep learning has emerged as a powerful tool for solving problems that were considered to be difficult in the past. It has
demonstrated impressive results on tasks like object recognition~\cite{donahue14,he2016deep}, facial
recognition~\cite{parkhi2015deep,sun2014deep}, speech processing~\cite{pmlrv48amodei16}, and machine translation~\cite{bahdanau2014neural}.
While many of these tasks are also important on mobiles and the Internet of Things (IoT), existing solutions are often
computation-intensive and require a large amount of resources for the model to operate. As a result, performing deep
inference\footnote{Inference in this paper refers to apply a pre-trained model on an input to obtain the corresponding output. This is
different from statistical inference.} on embedded devices can lead to long runtimes and the consumption of abundant amounts of resources,
including CPU, memory, and power, even for simple tasks~\cite{CanzianiPC16}. Without a solution,
 the hoped-for advances on embedded sensing will not arrive.


Numerous approaches have been proposed to accelerate deep inference on embedded devices. These include designing specialize hardware to
reduce the computation or memory latency~\cite{georgiev2017low}, compressing a pre-trained model to reduce its storage and memory footprint
as well as computational requirements~\cite{han2016eie}, and offloading some, or all, computation to a cloud
server~\cite{Kang2017neurosurgeon,teerapittayanon2017distributed}. Compared to specialized hardware, software-based model compression
techniques have the advantage of being readily deployable on commercial-off-the-self hardware; and compared to computation offloading,
model compression enables on-device inference which in turn allows faster response time and has less privacy concerns. Such advantages make
model compressions attractive on existing hardware platforms where computation offloading is not feasible.


However, model compression is not a free lunch as it comes at the cost of loss in prediction accuracy~\cite{cheng2017survey}. This means that one must
carefully choose the model compression technique and its parameters to effectively trade precision for computation and resource
requirements or less energy consumption. Furthermore, as we will show in this paper, the reduction in the model size does not necessarily
translate into faster inference time. Because a model compression technique is not always beneficial, it is important to understand when
and how to apply it.

In this paper, we aim to understand deep learning model compression techniques for embedded inference. This knowledge not only informs the
better deployment of computation-intensive models, but also enables destining more efficient architectures for models and hardware for
mobile and IoT devices.

To that end, we develop a quantitative approach  to characterize two mainstream model compression techniques,
pruning~\cite{cheng2017survey} and data quantization~\cite{Gong2014Compressing}. We apply the techniques to the image classification
domain, an area where deep learning has made great breakthroughs and a rich set of pre-trained models are available. We evaluate the
compression results on the NVIDIA Jetson TX2 embedded deep learning platform and consider a wide range of influential DNN models using the
50K images from the ImageNet ILSVRC 2012 validation dataset~\cite{imagenet2012}.


We show that while there is significant gain for choosing the right compression technique and parameters, mistakes can seriously hurt the
performance. We quantify how different model compression techniques and parameters affect the inference time, energy consumption, model
storage requirement and the prediction accuracy. As a result, our work provides insights on when and how to apply deep learning model
compression techniques on embedded devices.

The main contributions of this paper are two folds:

\begin{itemize}
\item We present an extensive study to characterize and understand how two popular model compression techniques perform on a
    representative embedded deep learning platform;
\item Our work offers insights on when and how to apply compression techniques for embedded deep inference.
\end{itemize}
