\section{Related Work}
%Our work is closely related to work in web browsing optimization, task scheduling, energy optimization and predictive modeling.

Numerous techniques have been proposed to optimize web browsing, through e.g.
prefetching~\cite{wang2012far} and caching~\cite{qian2012web} web contents, 
or re-constructing the browser workflow~\cite{zhao2015energy,mai2012case} or the TCP protocol~\cite{Xie:2017:AMW:3081333.3081367}. Most of the
prior work target homogeneous systems and do not optimize across networking environments. The work presented by Zhu \emph{et
al.}~\cite{YZhu13} and prior work~\cite{ren2016optimise} were among the first attempts to optimize web browsing on heterogeneous mobile
systems. Both approaches use statistical learning to estimate the optimal configuration for a given web page. However, they do not consider
the impact of the networking environment, thus miss massive optimization opportunities. Bui \emph{et al.}~\cite{bui2015rethinking} proposed
several web page rendering techniques to reduce energy consumption for mobile web browsing. Their approach uses analytical models to
determine which processor core (big or little) to use to run the rendering process. The drawback of using an analytical model is that the
model needs to be manually re-tuned for each individual platform to achieve the best performance. Our approach avoids the pitfall by
developing an approach to automatically learn how to best schedule rendering process. 



%Machine learning based predictive
%modeling has been proven to be effective at learning
%how to reduce the system overhead\cite{shye2009into,berral2011adaptive,kan2012eclass}.
%Shye et al.\cite{shye2009into} proposed a linear regression model
%that accurately predict the power consumption of a mobile architecture to
%reduce the overhead from screen and
%CPU.  J. L. Berral et al.\cite{berral2011adaptive} used the machine learning
%to estimate the resource usage and task service-level-agreement to
%to find good task scheduling that balance the revenue for executed tasks, quality of
%service, and power consumption in the data center.
%Kan et al.\cite{kan2012eclass} presented
%an energy-aware frequency assignment algorithm for DVFS
%processors, which is a classification-based approach to predict the optimal frequency for next interval.
%And Z. Tang et al.\cite{tang2015energy} described a machine learning based online scheduling
%that reduces the tail energy by deferring transmissions.
%However, none of the previous
%research in predictive modeling based application optimization
%addresses the problem of multi-task scheduling on a
%heterogeneous platform.
%The Zhu\cite{YZhu13} used off-line profiling to create a
%regression model that is employed to predict the webpage load time
%and energy, then map the webpage to the lower energy configuration
%that the load time within a limit value.
%Unlike Zhu, our approach uses SVM to predict the best configuration directly,
%and map the render process depends on what metrics to use.
