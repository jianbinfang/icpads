\documentclass[conference,compsoc]{IEEEtran}
\usepackage{booktabs} % For formal tables
\usepackage{graphicx}
\usepackage{subfig}
\usepackage{blindtext}
\usepackage{adjustbox}
\usepackage{multirow}
\usepackage{color}
\usepackage{booktabs}
\usepackage{tabularx}
\usepackage{colortbl}

\usepackage{tikz}
\usepackage{listings}
\usepackage{etoolbox}
\usepackage{subfig}
\usepackage{url}
\usepackage{setspace}
\usepackage[british,english]{babel}
\usepackage{color, colortbl}
\usepackage{enumitem}
\usepackage{balance}
%\usepackage{hyperref}

\definecolor{Gray}{gray}{0.9}
\linespread{0.9}

\newcommand{\circled}[2][]{\tikz[baseline=(char.base)]
	{\node[shape = circle, draw, inner sep = 1pt]
		(char) {\phantom{\ifblank{#1}{#2}{#1}}};%
		\node at (char.center) {\makebox[0pt][c]{#2}};}}
\robustify{\circled}

\newcommand{\ProfileSize}{100MB\xspace}
\newcommand{\PCA}{\texttt{PCA}\xspace}
\newcommand{\cs}{\texttt{cs}\xspace}
\newcommand{\KNN}{\texttt{KNN}\xspace}
\newcommand{\SVM}{\texttt{SVM}\xspace}
\newcommand{\SVMs}{\texttt{SVMs}\xspace}
\newcommand{\NB}{\texttt{NB}\xspace}
\newcommand{\LR}{\texttt{LR}\xspace}
\newcommand{\ANN}{\texttt{ANN}\xspace}
\newcommand{\PCs}{\texttt{PCs}\xspace}
\newcommand{\PC}{\texttt{PC}\xspace}
\newcommand{\RDD}{\texttt{RDD}\xspace}
\newcommand{\RDDs}{\texttt{RDDs}\xspace}
\newcommand{\MLP}{\texttt{MLP}\xspace}
\newcommand{\Quasar}{\textsc{Quasar}\xspace}
\newcommand{\DColocation}{\textsc{Pairwise}\xspace}
\newcommand{\Oracle}{\texttt{oracle}\xspace}
\newcommand{\oracle}{\texttt{oracle}\xspace}
\newcommand\FIXME[1]{\textcolor{red}{FIX:}\textcolor{red}{#1}}

\newcommand{\BBCW}{\texttt{news.bbc.co.uk}\xspace}

\newcommand{\WS}{\texttt{WS}\xspace}
\newcommand{\SML}{\texttt{S-ML}\xspace}

\newcommand{\DOM}{\texttt{DOM}\xspace}

\newcommand{\EDP}{\texttt{EDP}\xspace}
\newcommand{\Ondemand}{\texttt{ondemand}\xspace}
\newcommand{\Performance}{\texttt{performance}\xspace}
\newcommand{\Interactive}{\texttt{interactive}\xspace}
\newcommand{\Conservative}{\texttt{conservative}\xspace}
\newcommand{\Powersave}{\texttt{powersave}\xspace}
\newcommand{\bfp}{\texttt{best-performing}\xspace}

\newcommand\cparagraph[1]{\vspace{1.5mm} \noindent \textbf{#1}\xspace}

\usepackage{multirow}

\begin{document}
%
% paper title
% Titles are generally capitalized except for words such as a, an, and, as,
% at, but, by, for, in, nor, of, on, or, the, to and up, which are usually
% not capitalized unless they are the first or last word of the title.
% Linebreaks \\ can be used within to get better formatting as desired.
% Do not put math or special symbols in the title.
\title{Network Delay-Aware Optimization for Web Browsing \\on Heterogeneous Mobile Platforms\vspace{-4 mm}}

 
\author{\IEEEauthorblockN{Jie Ren\IEEEauthorrefmark{2}, Xiaoming Wang\IEEEauthorrefmark{2}, Feng Tian\IEEEauthorrefmark{2},  Hai Wang\IEEEauthorrefmark{3},
 Jie Zheng\IEEEauthorrefmark{3}, Zheng Wang\IEEEauthorrefmark{1}} \IEEEauthorblockA{
\IEEEauthorrefmark{2} Shaanxi Normal University, China, \IEEEauthorrefmark{3}Northwest University, China, \IEEEauthorrefmark{1}Lancaster University, UK \\
\IEEEauthorrefmark{2}\{renjie, wangxm, tianfeng\}@snnu.edu.cn, \IEEEauthorrefmark{3}\{hwang, jzheng\}@nwu.edu.cn,
\IEEEauthorrefmark{1}z.wang@lancaster.ac.uk \vspace{-4 mm} } }

% make the title area
\maketitle

% As a general rule, do not put math, special symbols or citations
% in the abstract
\begin{abstract}
In this paper, we propose a machine
learning based approach to predict which of the heterogeneous processors to use to render the web content and the operating frequencies of heterogeneous processors.
We do so by first learning, \emph{offline}, a set of predictive models for a range of networking
environments. We then choose a learnt model at runtime to predict the optimal processor configuration. The prediction is based on the web content, the network status and the optimization goal.
We obtain, on average, over 21\% and 39\%
of improvement respectively for load time and energy consumption, when compared to two competitive
optimisers that are specifically tuned for mobile web browsing.
\end{abstract}





% For peer review papers, you can put extra information on the cover
% page as needed:
% \ifCLASSOPTIONpeerreview
% \begin{center} \bfseries EDICS Category: 3-BBND \end{center}
% \fi
%
% For peerreview papers, this IEEEtran command inserts a page break and
% creates the second title. It will be ignored for other modes.
\IEEEpeerreviewmaketitle

\section{Introduction}
In recent years, deep learning has emerged as a powerful tool for solving problems that were considered to be difficult in the past. It has
demonstrated impressive results on tasks like object recognition~\cite{donahue14,he2016deep}, facial
recognition~\cite{parkhi2015deep,sun2014deep}, speech processing~\cite{pmlrv48amodei16}, and machine translation~\cite{bahdanau2014neural}.
While many of these tasks are also important on mobiles and the Internet of Things (IoT), existing solutions are often
computation-intensive and require a large amount of resources for the model to operate. As a result, performing deep
inference\footnote{Inference in this paper refers to apply a pre-trained model on an input to obtain the corresponding output. This is
different from statistical inference.} on embedded devices can lead to long runtimes and the consumption of abundant amounts of resources,
including CPU, memory, and power, even for simple tasks~\cite{CanzianiPC16}. Without a solution,
 the hoped-for advances on embedded sensing will not arrive.


Numerous approaches have been proposed to accelerate deep inference on embedded devices. These include designing specialize hardware to
reduce the computation or memory latency~\cite{}, compressing a pre-trained model to reduce its storage and memory footprint as well as
computational requirements~\cite{}, and offloading some, or all, computation to a cloud
server~\cite{Kang2017neurosurgeon,teerapittayanon2017distributed}. Compared to specialized hardware, software-based model compression
techniques have the advantage of being readily deployable on commercial-off-the-self hardware; compared to computation offloading, model
compression enables on-device inference which in turn allows faster response time and has less privacy concerns. These advantages make
model compressions attractive on existing hardware platforms where computation offloading is not feasible.


However, model compression is not a free lunch as it comes at the cost of loss in prediction accuracy~\cite{}. This means that one must
carefully choose the model compression technique and its parameters to effectively trade precision for computation and resource
requirements or less energy consumption. Furthermore, as we will show in this paper, the reduction in the model size does not necessarily
translate into faster inference time. Because a model compression technique is not always beneficial, it is important to understand when
and how to apply it.

In this paper, we aim to understand deep learning model compression techniques for embedded inference. This knowledge allows not only the
better deployment of computation-intensive models, but also designing more efficient architectures for models and hardware for on mobile
and IoT devices.

In this work, we develop a quantitative approach  to characterize two mainstream model compression techniques, pruning~\cite{} and data
quantization~\cite{}. We apply the techniques to the image classification domain, an area where deep learning has made impressive
breakthroughs and a rich set of pre-trained models are available. We evaluate the compression results on the NVIDIA Jetson TX2 embedded
deep learning platform and consider a wide range of influential DNN models using the 50K images from the ImageNet ILSVRC 2012 validation
dataset~\cite{}.

We show that while there is significant gain for choosing the right compression technique and parameters, mistakes can seriously hurt the
performance. We quantify how different model compression techniques and parameters affect the inference time, energy consumption, model
storage requirement and the prediction accuracy. As a result, our work provides insights on when and how to apply deep learning model
compression techniques on embedded devices.

The main contributions of this paper are two folds:

\begin{itemize}
\item We present an extensive study to characterize and understand how two popular model compression techniques perform on a
    representative embedded deep learning platform;
\item Our work offers insights on when and how to apply compression techniques for embedded deep inference.
\end{itemize}

\section{Motivation}


%The JavaScript engine's performance that involves the compiler,
%garbage collector, etc., and the final display work on GPU are separate issues beyond the scope of this work.

\begin{table}[!t]
\caption{The best-performing available governor}
\vspace{-3mm}
\scriptsize
\begin{center}
        \begin{tabular}{llll}
        \toprule
        &\textbf{Load time}&\textbf{Energy}&\textbf{EDP}\\
        \midrule
            Regular 3G                     &\Performance&\Powersave&\Powersave\\
            \rowcolor{Gray}Regular 4G                     &\Performance&\Conservative&\Interactive\\
            WiFi            &\Interactive&\Ondemand&\Interactive\\
        \bottomrule
        \end{tabular}
\end{center}
\label{tab:best-governor}
\vspace{-5mm}
\end{table}


\begin{figure}[!t]
	\centering
	\subfloat[][Load time]{\includegraphics[width=0.22\textwidth]{figure/laod4pagesloadtime.pdf}}
    \hfill
    \subfloat[][Energy consumption]{\includegraphics[width=0.22\textwidth]{figure/load4pagesEnergy.pdf}}
    \vspace{-2mm}
    \caption{The achieved total load time (a), energy consumption (b)  when a user was browsing four news pages from \BBCW.
    We show the results for \Oracle, the \bfp existing CPU frequency governor, and \Interactive in three typical networking environments. There is significant room for improvement. }
    \vspace{-5mm}
    \label{fig:motivation}
\end{figure}



Consider a scenario for browsing four \texttt{BBC} news pages, starting from the home page of \BBCW. Our evaluation platform has  a
Cortex-A15 (big) and a Cortex-A7 (little) processors.

\vspace{-1mm}
\cparagraph{Networking Environments.} We consider three typical networking
environments: Regular 3G, Regular 4G and WiFi (see Section~\ref{sec:networks} for more details). To ensure reproducible results, web requests and responses are deterministically replayed by the client and a web server respectively. The
web server simulates the download speed and latency of a given network setting, and we record and deterministically replay the
user interaction trace for each testing scenario.


\vspace{-1mm}
\cparagraph{Scheduling Strategies.} We schedule the Chromium rendering engine (i.e., \texttt{CrRendererMain}) to run on either the big or the little
core under different clock frequencies to find the best processor configuration per test case. We refer this best-found configuration as
the \Oracle because it is the best performance we can get via CPU frequency scaling and task mapping. We use the \Interactive CPU
frequency governor as the baseline, which is the default frequency governor on many mobile devices \cite{Seo2015Big}. We also compare with the best-performing governor found from mainstream CPU governors,
including
the \Interactive and other four strategies: \Performance, \Conservative, \Ondemand and \Powersave.


 %We evaluate the performance of each strategy in three typical networking environments, Regular 3G, Regular 4G,
%and WiFi, where a typical smartphone user would  experience. Table~\ref{tab:network} shows the configuration of each networking environment setting.
%Later in this paper, we evaluate our approach on a wider range of networking environments (see Section~\ref{}).

\cparagraph{Motivation Results.} Table~\ref{tab:best-governor} lists the best-performing governor chosen from the five existing CPU
frequency governors, and Figure~\ref{fig:motivation} summarizes the performance of each strategy for each optimization metric. While
\Interactive  gives the best \EDP compared to other existing governors in a Regular 4G and a WiFi environments, it fails to deliver the
best-available performance for load time and energy consumption. Furthermore, there is significant room for improvement for the best-performing
existing governor when compared to the \Oracle.  On average, the \Oracle outperforms the best-performing governor by 154.6\%,
70.6\% respectively for load time and energy consumption across networking environments.
More importantly, the oracle processor configuration varies across web pages, networking
environments and evaluation metrics -- no single configuration consistently delivers the best-available performance.

\cparagraph{Lessons Learned.} This example shows that the current mainstream CPU frequency governors are ill-suited for mobile web browsing
and the best processor configuration depends on the network and the optimization goal. There is a need for a better scheduler that can
adapt to the webpage workload, the networking environment and the optimization goal. In the remainder of this article, we describe such an
approach based on machine learning.

\section{Overview of our approach}







\section{Predictive Modeling}

Our models for processor configuration prediction are a set of Support Vector Machines (\SVMs)~\cite{vapnik1998statistical}. We use the
Radial basis kernel because it can model both linear and non-linear classification problems. We use the same methodology to learn all
predictors for the target networking environments and optimization goals (i.e., load time, energy consumption, and \EDP).


\subsection{Network Monitoring and Characterization}
\label{sec:networks}
\begin{table}[!t]
\caption{Networking environment settings}
\vspace{-3mm}
\scriptsize
\begin{center}
        \begin{tabular}{lp{2cm}p{2.5cm}c}
        \toprule
        &\textbf{Uplink bandwidth}&\textbf{Downlink bandwidth}&\textbf{Delay}\\
        \midrule
            \rowcolor[gray]{.92}Regular 2G                     &50kbps&100kbps&1000ms\\
            Good 2G                             &150kbps&250kbps&300ms\\
            \rowcolor[gray]{.92}Regular 3G      &300kbps&550kbps&500ms\\
            Good 3G                             &1.5Mbps&5.0Mbps&100ms\\
            \rowcolor[gray]{.92}Regular 4G      &1.0Mbps&2.0Mbps&80ms\\
            Good 4G                             &8.0Mbps&15.0Mbps&50ms\\
            \rowcolor[gray]{.92}WiFi            &15Mbps&30Mbps&5ms\\
        \bottomrule
        \end{tabular}
\end{center}
\label{tab:networkDetails}
\vspace{-5mm}
\end{table}

\begin{figure}[!t]
\begin{center}
\includegraphics[width=0.45\textwidth]{figure/network.pdf}
\end{center}
\vspace{-2mm}
\caption{Webpage rendering time w.r.t. content download time when using the \Interactive governor.}
\vspace{-4mm}
\label{fig:latency}
\end{figure}


The communication network has a significant impact on the web rendering strategy. 
Table~\ref{tab:networkDetails} lists the networking environments considered in this work. The settings and categorizations are based on the
measurements given by an independent study~\cite{opensignalUK}.  Figure~\ref{fig:latency} shows the webpage rendering time with respect to
the download time under each networking environment when using the \Interactive governor.  The download
time dominates the end to end turnaround time for a 2G and a Regular 3G environments; and by contrast, the rendering time accounts for most
of the turnaround time for a Good 4G and a WiFi environments when the delay is small.


In this work, we learn a predictor per optimization goal for each of the seven networking environments. 
To determine which network environment
the user is currently in, we develop a lightweight network monitor to measure the network bandwidths and delay between the web server and
the device. The network monitor utilizes the communication link statistics that are readily available on commodity smartphones. Measured
data are averaged over the measurement window. The
measurements are then used to map the user's networking environment to one of the pre-defined settings in Table~\ref{tab:networkDetails},
by finding which of the settings is closest to the measured values. The closeness or distance, $d$, is calculated using the following
formula: \vspace{-1mm}
\begin{equation}
d = \alpha |db_{m} - db| + \beta |ub_{m} - ub| + \gamma |d_{m} - d|
\label{eq:cat}
\end{equation}

where $db_m$, $ub_m$, and $d_m$ are the measured downlink bandwidth, upload bandwidth and delay respectively, $db$, $ub$, and $d$ are the
downlink bandwidth, upload bandwidth and delay of a network category, and $\alpha$, $\beta$, $\gamma$ are weights. The weights are
automatically learned from the training data, with an averaged value of 0.3, 0.1 and 0.6 respectively for $\alpha$, $\beta$, and $\gamma$.


\subsection{Training the Predictor}



%\begin{table}[!t]
%\caption{Optimal processor configurations for web rendering under Regular 3G and WiFi network environments}
%\scriptsize
%\begin{center}
%        \begin{tabular}{lccccccccc}
%        \toprule
%        \rowcolor[gray]{.92}& \multicolumn{2}{c}{Load time} &\multicolumn{2}{c}{Energy}& \multicolumn{2}{c}{EDP} \\
%        & A15 & A7 & A15 & A7 & A15 & A7\\
%        \midrule
%             \multirow{4}{*}{Regular 3G}     &\fcolorbox{white}{lightgray}{1.6}&0.4   &0.4&\fcolorbox{white}{lightgray}{0.4}     &\fcolorbox{white}{lightgray}{0.8}&0.4\\
%                            &\fcolorbox{white}{lightgray}{1.7}&0.8   &\fcolorbox{white}{lightgray}{0.4}&0.4     &\fcolorbox{white}{lightgray}{0.8}&0.8\\
%                            &\fcolorbox{white}{lightgray}{1.8}&0.8   &\fcolorbox{white}{lightgray}{0.8}&0.4     &\fcolorbox{white}{lightgray}{0.4}&0.4\\
%                            &\fcolorbox{white}{lightgray}{1.9}&0.8    &0.8&\fcolorbox{white}{lightgray}{0.8}     &-&- \\
%        \midrule
%             \multirow{4}{*}{WiFi}          &\fcolorbox{white}{lightgray}{1.6}&0.4    &\fcolorbox{white}{lightgray}{0.8}&0.4     &\fcolorbox{white}{lightgray}{1.2}&0.8\\
%                            &\fcolorbox{white}{lightgray}{1.7}&0.8   &\fcolorbox{white}{lightgray}{0.8}&0.8     &\fcolorbox{white}{lightgray}{1.2}&0.4\\
%                           &\fcolorbox{white}{lightgray}{1.8}&0.8    &\fcolorbox{white}{lightgray}{1.2}&0.4     &\fcolorbox{white}{lightgray}{0.8}&0.8\\
%                            &\fcolorbox{white}{lightgray}{1.9}&0.8    &-&-                                      &-&-      \\
%        \bottomrule
%        \end{tabular}
%\end{center}
%\label{tab:bestConfig}
%\vspace{-5mm}
%\end{table}

The training process involves finding the best processor configuration and extracting feature values for each training webpage,
and learn a model from the training data.


\cparagraph{Generate Training Data.}
%Our predictor is built \emph{offline} using a set of training webpages.
In this work, we used around 900 webpages to train a \SVM predictor; we then evaluate the learnt model on the other 100 unseen webpages.
These training webpages are selected from the landing page of the top 1000 hottest websites ranked by \texttt{www.alexa.com}. We use Netem~\cite{hemminger2005network}, a Linux-based network enumerator, to emulate
various networking environments to generate the training data. We exhaustively execute the rendering
engine under different processor settings and record the optimal configuration for each optimization goal and each networking environment.
We give each optimal configuration a unique label. For each webpage, we also extract values of a set of selected features.

%For example, Table~\ref{tab:bestConfig} lists the processor configurations for 3G and WiFi network environments, where the gray box indicates
%which core to use to run the rendering process.

\cparagraph{Building The Model.} The feature values together with the labeled processor configuration are supplied to a supervised learning
algorithm~\cite{kotsiantis2007supervised}. The learning algorithm tries to find a correlation from the feature values to the optimal
configuration and produces a \SVM model per networking environment per optimization goal. Because we target two optimization metrics and
seven networking environments, we have constructed 14 \SVM models in total. 

\subsection{Web Features \label{sec:web_features}}
One of the key aspects in building a successful predictor is finding the right features to characterize the input workload. In this work,
we consider a set of features extracted from the web contents. These features are collected by our feature extraction pass. To gather the
feature values, the feature extractor  first obtains a reference for each \DOM element by traversing the \DOM tree and then uses the
Chromium API, \texttt{document.getElementsByID}, to collect node information. We started from 214 raw features, including the number of
\DOM nodes, HTML tags and attributes of different types, and the depth of the \DOM tree, etc. All these features can be collected at
runtime from the browser. The types of the raw features are given in Table~\ref{tab:rawfeature}. These features are selected based
on our intuition and prior work~\cite{ren2016optimise,nejati2016depth,asrese2016wepr}.
It is important to note that the
collected feature values are encoded to a vector of real values.



\begin{figure}[!t]
	\centering
	\subfloat[][Principal components]{\includegraphics[width=0.23\textwidth]{figure/pca.pdf}}
    \hfill
    \subfloat[][Top 7 most important features]{\includegraphics[width=0.23\textwidth]{figure/pcacontri.pdf}}
    \vspace{-2mm}
    \caption{The percentage of principal components (\PCs) to the overall feature variance (a), and contributions of the seven most important
     raw features in the \PCA space (b).}
    \label{fig:pca}
    \vspace{-4mm}
\end{figure}



\cparagraph{Feature Reduction.} To improve the generalization ability of our models, i.e., reducing the likelihood of over-fitting on our training data, we reduce some features
through applying Principal Component Analysis (\PCA)~\cite{dunteman1989principal} to the raw feature
space. \PCA transforms the original inputs into a set of principal components (\PCs) that are linear combinations of the inputs. After
applying \PCA to the 214 raw features, we choose the top 18 principal components (\PCs) which account for around 95\% of the variance of
the original feature space. We record the \PCA transformation matrix and use it to transform the raw features of the new webpage to \PCs
during runtime deployment. Figure ~\ref{fig:pca}a illustrates how much feature variance that each component accounts for. This figure shows
that predictions can accurately draw upon a subset of aggregated feature values.

\cparagraph{Feature Normalization.} Before passing our features to a machine learning model we need to scale each of the features to a
common range (between 0 and 1) in order to prevent the range of any single feature being a factor in its importance. Scaling features does
not affect the distribution or variance of their values. To scale the features of a new webpage during deployment we record the minimum and
maximum values of each feature in the training dataset, and use these to scale the corresponding features.

\cparagraph{Feature Analysis.} To understand the usefulness of each raw feature, we apply the Varimax rotation~\cite{manly2016multivariate}
to the \PCA space. This technique quantifies the contribution of each feature to each \PC. Figure~\ref{fig:pca}b shows the top 7 dominant
features based on their contributions to the \PCs. Features like the webpage size and the number of \DOM nodes are most important,
because they strongly correlate to the download time and the complexity of the webpage. Other features like the depth of the \DOM tree, and
the numbers of different attributes and tags, are also useful, because they determine how the webpage should be presented
and how do they correlate to the rendering cost. 

%Using this technique, we sort the raw features and list top 10 in Table ~\ref{tab:selected_features} according to the importance.

\begin{table}[t!]
\caption{Raw web feature categories}
\vspace{-2mm}
\small
\centering
        \begin{tabular}{rll}
        \toprule
        \multirow{2}{*}{DOM Tree} & \#DOM nodes & depth of tree \\
                & \#each HTML tag & \#each HTML attr. \\
        \rowcolor[gray]{.92}Other  & size of the webpage (Kilobytes) & \\
        \bottomrule
        \end{tabular}
\label{tab:rawfeature}
\vspace{-2mm}
\end{table}

%\begin{table}[!t]
%\caption{Selected web features}
%\small
%\centering
%        \begin{tabular}{rlrl}
%        \toprule
%                            1&size of the webpage    &6&\#Tag.script\\
%        \rowcolor[gray]{.92}2&\#DOM nodes            &7&\#Tag.img  \\
%                            3&depth of tree          &8&\#Attr.content \\
%        \rowcolor[gray]{.92}4&\#Attr.style           &9&\#Tag.img\\
%                           5&\#Tag.link            &10&\#Attr.media \\
%        \bottomrule
%        \end{tabular}
%\label{tab:selected_features}
%\vspace{-2mm}
%\end{table}






%Figure~\ref{fig:example} compares the resultant performance of our model against two widely used CPU frequency governors: \Powersave and
%\Interactive. The x-axis of the diagram shows the load time for the three webpages and the y-axis shows the CPU frequency chosen by each
%scheme. We also give the power consumption of each scheme.



\begin{figure*}[!t]
\centering
\subfloat[][Model size]{\includegraphics[width=0.33\textwidth]{figure/quan_size.pdf}}
\hfill
\subfloat[][Inference time]{\includegraphics[width=0.33\textwidth]{figure/quan_time2.pdf}}
\hfill
\subfloat[][Accuracy]{\includegraphics[width=0.32\textwidth]{figure/quan_acc.pdf}}
\hfill
\subfloat[][Power consumption]{\includegraphics[width=0.33\textwidth]{figure/quan_power2.pdf}}
\hfill
\subfloat[][Energy consumption]{\includegraphics[width=0.33\textwidth]{figure/quan_energy2.pdf}}
\hfill
\subfloat[][precision, recall and F1 score]{\includegraphics[width=0.33\textwidth]{figure/quan_prf2.pdf}}
\hfill

\caption{The achieved model size (a) inference time (b) accuracy (c) power consumption (d)
energy consumption (e) and precision, recall and F1 score (e) before and after the compression by \quantization.
The compression technique to use depends on the optimization target.}
\label{fig:analy_quan}
\end{figure*}


\begin{figure*}[!t]
\centering
\subfloat[][Model size]{\includegraphics[width=0.33\textwidth]{figure/prun_size.pdf}}
\hfill
\subfloat[][Inference time]{\includegraphics[width=0.3\textwidth]{figure/prun_time.pdf}}
\hfill
\subfloat[][Accuracy]{\includegraphics[width=0.3\textwidth]{figure/top1_5_prun.pdf}}
\hfill
\subfloat[][Power consumption]{\includegraphics[width=0.3\textwidth]{figure/prun_power.pdf}}
\hfill
\subfloat[][Energy consumption]{\includegraphics[width=0.3\textwidth]{figure/prun_energy.pdf}}
\hfill
\subfloat[][precision, recall and F1 score]{\includegraphics[width=0.3\textwidth]{figure/prun_prf.pdf}}
\hfill

\caption{The change of the model size (a), inference time (b), accuracy/BLEU (c), power (d), energy consumption (e), and accuracy (e)
before and after applying \pruning.} \label{fig:analy_prun}
\end{figure*}

\section{Experimental Results}


\subsection{Roadmap}
Our experiments try to answer the following questions:

\begin{itemize}
\item bla
\item bla2
\item bla3
\end{itemize}

\subsection{Impact on the Model Storage Size}
Reducing the model storage size is crucial for embedded and IoT systems which often have a limited storage space. A smaller model size also
translates to smaller runtime memory footprint of less RAM space consumption. Figures~\ref{fig:analy_quan} and  \ref{fig:analy_prun}
illustrate how the different compression techniques and parameters affect the resulting model size.

As can be seen from Figure~\ref{fig:analy_quan} a, using an 8-bit data quantization can significantly reduce the model storage size,
leading to an average reduction of 74.5\%. From Figure~\ref{fig:analy_prun} a, we see that by removing some of the pathways of the neural
network, \pruning can also reduce the model size, although the gain is smaller than \quantization. On average, \pruning reduces the model
size by 27.2\% (\FIXME{xx MB}). An interesting observation is that, \pruning is particularly effective for obtaining a compact model for
NMT, an \RNN, with a reduction of 60\% on the model size. This is because there are typically many repetitive pathways in an \RNN due to
the natural of the network architecture. As we will discuss later, \pruning only leads to a minor degradation in the prediction accuracy
for NMT. This suggests that \pruning can be an effective model compression technique for \RNNs.



\subsection{Impact on Accuracy Metrics}
When compressing a model, we still want to largely maintain the performance of a compressed model. Therefore, it is importance to
effectively trade precision for storage space. Results in Figure~\ref{} compare how the prediction accuracy metrics are affected by model
compression.

We see that the sweat spot of \quantization depends on the neural network structure. Although an 8-bit representation leads to a minor
decrease in the prediction accuracy, a further reduction of a 6-bit representation is only profitable for xx networks.

\subsection{Impact on Inference Time}


\subsection{Impact on the Energy Consumption}


\subsection{Memory footprint}

\begin{figure}[!t]
\centering
\subfloat[][\quantization]{\includegraphics[width=0.4\textwidth]{figure/quan_mem.pdf}}
\hfill
\subfloat[][\pruning]{\includegraphics[width=0.4\textwidth]{figure/prun_mem.pdf}}
\hfill

\caption{Memory footprint before and after the compression by \quantization(a) and \pruning (b).}
\label{fig:footprint}
\end{figure}

\section{Related Work}
\DNNs have shown astounding successes in various tasks that previously seemed difficult~\cite{cho2014learning}. Despite the fact that many embedded
devices require precise sensing capabilities, adoption of \DNN models on such systems has notably slow progress. This mainly due to
\DNN-based inference being typically a computation intensive task, which inherently runs slowly on embedded devices due to limited
resources.

There has been a significant amount of work on reducing the storage and computation work by model compression.
For parameter sharing, Song et al.~\cite{han2015deep} proposed a method which quantizes the weights to enforce weight sharing and apply Huffman coding to compress model size. 
This method reduced the storage required by AlexNet by 35x and the size of VGG\_16 by 49x. In~\cite{Gong2014Compressing}, 
the authors tackled model storage issue by investigating information theoretical vector quantization methods for compressing the parameters of CNNs.

With respect to pruning weights, ~\cite{Li2016Pruning} presented an acceleration method for CNNs, in which the authors pruned filters from CNNs that are identified as having a small effect on the output accuracy. By removing whole filters in the network together with their connecting feature maps, the computation costs are reduced significantly. Meantime, a LASSO regression based channel selection and least square reconstruction were presented by~\cite{he2017channel}, this method was applied to accelerate very deep convolutional neural networks and obtained promising results.

Besides parameter sharing and pruning, the Low-rank factorization and Knowledge Distillation are also widely adopted in model compression. Denton et al.~\cite{denton2014exploiting}exploited the linear structure present within the convolutional filters to derive approximations that significantly reduce the required computation. ~\cite{lebedev2014speeding} proposed a simple two-step approach for speeding up convolution layers within large convolutional neural networks based on tensor decomposition and discriminative fine-tuning. A simple methodology to include a noise-based regularizer while training the student from the teacher was proposed in ~\cite{Sau2016Deep}, which provides a healthy improvement in the performance of the student network. Also, the same idea was proposed by ~\cite{Hinton2015Distilling}, which improved the acoustic model of a heavily used commercial system by distilling the knowledge in an ensemble of models into a single mode.



There is an extensive body of work on how to accelerate \DNN training using xx, xx, and xx. Our work aims to understand how to accelerate
deep learning inference by choosing the right model compression technique.


As an alternative to on-device inferencing, off-loading computation to the cloud can accelerate \DNN model inference
\cite{teerapittayanon2017distributed}. Neurosurgeon \cite{Kang2017neurosurgeon} identifies when it is beneficial (\eg in terms of energy
consumption and end-to-end latency) to offload a \DNN layer to be computed on the cloud. The Pervasive \CNN~\cite{song2017towards} generates
multiple computation kernels for each layer of a \CNN, which are then dynamically selected according to the inputs and user constraints. A
similar approach presented in \cite{servia2017personal} trains a model twice, once on shared data and again on personal data, in an attempt to
prevent personal data being sent outside the personal domain. Computation off-loading is not always applicable due to privacy, latency or
connectivity issues. The work presented by Ossia \etal partially addresses the issue of privacy-preserving when offloading \DNN inference
to the cloud ~\cite{osia2017hybrid}. Our work is complementary to prior work on computation off-loading by offering insights to choose the
optimal compression technique to best optimize local inference.



\section{Conclusions}

This paper has presented an automatic approach to optimize web rendering on heterogeneous mobile platforms, providing significant
improvement over existing web-content-aware schedulers. We show that it is crucial to exploit the knowledge of the communication network
and the web contents to make effective scheduling decisions. We address the problem by using machine learning to develop predictive models
to predict which processor core to use to run the web rendering process and the optimum frequency of the processors. As a departure from
prior work, our approach consider of the network status, web workloads and the optimization goals. Our techniques are implemented as an
extension in the Chromium web browser and evaluated on a representative big.LITTLE heterogeneous mobile platform using the top 1000 hottest
websites. Experimental results show that our approach achieves over 80\% of the \oracle performance, and outperforms the state-of-the-arts
by 1.21x and 1.39x for load time and energy consumption.

%\vspace{-2mm}


\bibliographystyle{IEEEtran}
\bibliography{IEEEabrv,refs}


% that's all folks
\end{document}


