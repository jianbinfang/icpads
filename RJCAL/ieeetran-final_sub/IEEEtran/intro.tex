\section{Introduction}
\vspace{-1mm}

Web is a major information portal on mobile devices~\cite{mobilestat}. However, web browsing is poorly optimized and continues to consume a
significant portion of battery power on mobile devices~\cite{thiagarajan2012killed,d2016energy,cao2017deconstructing}. Heterogeneous
multi-cores, such as the ARM big.LITTLE architecture~\cite{arm},  offer a new way for energy-efficient mobile computing. However, current
mobile web browsers rely on the operating system (OS) to exploit the heterogeneous cores. Since the OS has little knowledge of the web
workload and how does the network affect web rendering, its decision is often sub-optimal. This leads to poor energy
efficiency~\cite{zhu2015event}, draining the battery faster than necessary and irritating mobile users.

Rather than letting the OS make all the scheduling decisions by passively observing the system's load, our work enables the browser to
actively participate in decision making. Specifically, we want the browser to decide which heterogeneous core and the optimal processor
frequencies to use to run the rendering engine. We believe such a decision must consider the web content, the optimization goal, and how
the network affects the rendering process. We achieve this by employing machine learning to automatically build predictors based on
empirical observations gathered from a set of training examples. The trained models are then used at runtime to predict the optimal
processor configuration for any \emph{unseen} webpage.


The key contribution of this article is a novel machine learning based web rendering scheduler that can leverage knowledge of the network
and webpages to optimize mobile web browsing. We compare our approach against two state-of-art web browser
schedulers~\cite{YZhu13,ren2016optimise} on a representative ARM big.LITTLE mobile platform. Experimental results show that our approach
outperforms the state-of-the arts by delivering over 1.2x (up to 1.4x) improvement across evaluation metrics. Our techniques generally
applicable, as they are useful for not only web browsers but also a large number of mobile apps that are underpinned by web rendering
techniques~\cite{Charland}. Our work is open sourced\footnote{Code is available at: https://goo.gl/gmg4Pk}.
