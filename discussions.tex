\section{Discussions}
We have demonstrated that choosing the correct model compression setting can accelerate inference time and lower the computation resource
requirements. The right decision depends on the constraints on resources, time and precision. We now summarize what have learned from our
experiments and the potential future research directions.

Our evaluation reveals that \dquantization is particularly effective in reducing the model storage size and runtime memory footprint. As
such, it is attractive for devices with limited memory resources, particularly for small-formed IoT devices. However, \quantization leads
to longer inference time due to the overhead of the de-quantization process. Therefore, future research is needed to look at how to reduce
the overhead of de-quantization or eliminating de-quantization by finding alternative ways for precision recovery. We also observe that an
8-bit integer quantization seems to be a good trade-off between the model storage size and the precision. This strategy also enables SIMD
vectorization on CPUs and GPUs as multiple 8-bit scalar values can be packed into one vector register. Using less than 8 bits is less
beneficial on traditional CPUs and GPUs, but could be useful on FGPAs or specialized hardware with purpose-built registers and memory
load/store units. We believe studying when and how to apply \dquantization to a specific domain or a neural network architecture would be
an interesting research direction.

We empirically show that \pruning allows us to precisely control in the prediction precision. This is useful for applications like security
and surveillance where we need a degree of confidences on the predictive outcome. Compared to \dquantization, \pruning is less effective in
reducing the model storage size, and thus may require larger storage and memory space. We also find that \pruning is particularly effective
for \RNNs, perhaps due to the recurrent structures of an \RNN. This finding suggests \pruning can be an important method for accelerating
\RNN on embedded systems.

Combining \dquantization and \pruning is an interesting approach, as it can bring together the best part of both techniques (see
Section~\ref{sec:combin}). However, one must  make sure the overhead of \dquantization does not eclipse the reduction in inference time by
applying \pruning. One interesting research question could be: ``Can we find other data representations to better quantize a model?". For
examples, instead of using just integers, one can use a mixture of floating point numbers and integers with different bit widths – by
giving wider widths for more important weights. Furthermore, given that it is non-trivial to choose the right compression settings, it will
be very useful to have a tool to automatically search over the Pareto design space to find a good configuration to meet the conflict
requirements of the model size, inference time and prediction accuracy.  As a final remark of our discussion, we hope our work can
encourage a new line of research on auto-tuning of deep learning model compression.
