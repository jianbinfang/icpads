\begin{abstract}
The recent advances in deep neural networks (\DNNs) make them attractive for embedded systems. However, it can take a long time for DNNs
to make an inference on resource-constrained mobile and IoT devices. Model compression techniques can address the computation issue of
deep inference on embedded devices without requiring specialized hardware or computation-offloading that is often infeasible due to
privacy concerns, high latency, or the lack of connectivity. However, it remains unclear how model compression techniques perform across
a wide range of \DNNs. To design efficient embedded deep learning solutions, we need to understand their behaviors. This work develops a
quantitative approach to characterize model compression techniques on a representative embedded deep learning architecture, the NVIDIA
Jetson Tx2. We perform extensive experiments by considering 11 influential neural network architectures from the image classification and
the natural language processing domains. We experimentally show that how two mainstream compression techniques, data quantization and
pruning, perform on these network architectures and the implications of compression techniques to the model storage size, inference time,
energy consumption and performance metrics. We demonstrate that there are opportunities to achieve fast deep inference on embedded
systems, but one must carefully choose the compression settings. Our results provide insights on when and how to apply model compression
techniques and guidelines for designing efficient embedded deep learning systems.
\end{abstract}
